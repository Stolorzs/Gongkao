<script
  src="https://cdn.mathjax.org/mathjax/latest/MathJax.js?config=TeX-AMS-MML_HTMLorMML"
  type="text/javascript">
</script>

$$
% 文档类(模板)
\documentclass{ctexart}

% 汉字盒子宏包
\usepackage{hanzibox}

\title{\Large 用\texttt{hanzibox}宏包制作作文方格纸}
\author{寄存器}
\date{\today}

% 正文区(有且只能有一个)
\begin{document}

\maketitle

在语文教学中,往往需要在习题或试卷中排版方格纸以书写作文。
为此,在笔者开发的\texttt{hanzibox}宏包中添加了一个用于
制作作文方格纸的命令\verb|\writegrid|以按\texttt{hanzibox}%
宏包设置的外观样式生成生成指定行列数的作文方格纸。

\verb|\writegrid|命令仅需要一个必选参数用于指定方格纸的
行数,有两个专用的可选参数:(1)\texttt{gridcols}用于指定
列数;(2)\texttt{gridsepv}用于指定不同行的间距,间距值的
倒数是汉字格子高度的比例。

其它格子外观控制参数请参阅\texttt{hanzibox}说明书
(\texttt{texdoc hanzibox})。

\section{采用默认值绘制}
\begin{center}
  \writegrid{10}
\end{center}

\section{改变列数}
\begin{minipage}{0.45\textwidth}
  \centering
  \writegrid[gridcols=10]{10}
\end{minipage}\qquad
\begin{minipage}{0.45\textwidth}
  \centering
  \writegrid[gridcols=10]{10}
\end{minipage}

\section{改变间距}
% 可以用\hanziboxset命令设置参数
\hanziboxset{gridcols=10}
\begin{minipage}{0.45\textwidth}
  \centering
  \writegrid[gridsepv=3]{5}
\end{minipage}\qquad
\begin{minipage}{0.45\textwidth}
  \centering
  \writegrid[gridsepv=0.5]{5}
\end{minipage}

\section{改变格子样式}
% 可以用\hanziboxset命令设置参数
\hanziboxset{frametype=咪, framecolor=red,
             gridcols=10, charf=\Huge}
\centering
\writegrid[fillcolor=yellow!30]{5}

\bigskip

\centering
\writegrid[frametype=田, framecolor=blue]{5}

\end{document}

$$